\documentclass[a4j]{jarticle}

%% "%:"から始まる行を書くと「タグ行」として扱われる。Command+2でタグ行を挿入

\usepackage{mediabb} % pdfファイル挿入用
\usepackage{url}
\usepackage{graphicx}
\usepackage{ascmac}
\usepackage{here}
\usepackage{itembkbx}
\usepackage{subfigure} % for subfigure

\newenvironment{ite}{\begin{itemize}}{\end{itemize}} %箇条書き
\newenvironment{tab}{\begin{table}}{\end{table}} %表
\newenvironment{cen}{\begin{center}}{\end{center}} %中央揃え
\newenvironment{tabu}{\begin{tabular}}{\end{tabular}} %表
\newenvironment{bbox}{\begin{breakbox}}{\end{breakbox}} %四角囲い付け
\newenvironment{scr}{\begin{breakscreen}}{\end{breakscreen}} %丸囲い付け
\newenvironment{bit}{\begin{breakitembox}}{\end{breakitembox}} %囲い付き箇条書き, [c]等必要, {}に題名
\newenvironment{fig}{\begin{figure}}{\end{figure}} %図
\newenvironment{enu}{\begin{enumerate}}{\end{enumerate}} %箇条書き(数字)
\newenvironment{des}{\begin{description}}{\end{description}} %箇条書き(文字)
\newcommand{\esp}{echoserverプログラム}
\newcommand{\ecp}{echoclientプログラム}
\newcommand{\es}{echoserver}
\newcommand{\ec}{echoclient}
\newcommand{\sts}{simple\_talk\_serverプログラム}
\newcommand{\stc}{simple\_talk\_clientプログラム}

% 講義・演習名
\newcommand{\lectureName}{情報科学演習C}
% レポートの回数
\newcommand{\reportNumber}{2}
% レポートのタイトル
\newcommand{\reportTitle}{ネットワークプログラミング}
% 教員名
\newcommand{\teacherName}{小島 英春,内山 彰 教員}
% 課題締め切り日
\newcommand{\deadline}{2018年5月28日(月)  12:59}

% 提出者名
\newcommand{\studentName}{川上 遼太}
% 提出者所属
\newcommand{\studentAff}{基礎工学部 情報科学科 計算機科学コース 3年}
% 提出者学籍番号
\newcommand{\studentID}{09B16022}
% 提出者電子メールアドレス
\newcommand{\studentEmail}{u407773h@ecs.osaka-u.ac.jp}
% 課題提出日
\newcommand{\submitted}{2018年5月27日(日)}


\begin{document}

%Title page start
\begin{titlepage}
\mbox{\vspace{10cm}}

\begin{center}
{\Huge\bfseries
\lectureName \\
第\reportNumber{}回レポート} \\
\vspace{1cm}
{\Large \reportTitle}
\vspace{5cm}

{\large
\begin{tabular}{rcl}
担当教員 & : & \teacherName \\
提出者   & : & \studentName \\
所属/学年   & : & \studentAff\\
学籍番号 & : & \studentID \\
電子メール & : & \texttt{\studentEmail}\\
\\
提出日 & : & \submitted \\
締切日 & : & \deadline
\end{tabular}
}
\end{center}
\end{titlepage}
%Title page end

\tableofcontents
\newpage

\section{(課題2-1) echoclient.c}

この章ではTCPクライアントプログラムの作成について記述する.配布されたTCPサーバープログラムと作成したTCPクライアントプログラムは\ref{sec:echoserver_code},\ref{sec:echoclient_code}節に記載している.

\subsection{課題内容}

配布された\esp と通信することのできるクライアントプログラムを作成せよ.

\subsection{仕様}
\label{sec:echo_shiyo}

\ecp の仕様は以下の通りである.

\begin{ite}
\item 実行のコマンドは"echoclient host"と入力する.hostの部分には\esp を実行しているホスト名を指定すること.
\item \esp と\ecp はポート番号10120を介して接続する.
\item \ecp の標準入力から読み込んだ文字列を\esp に送信し,\esp が受け取った文字列を再度\ecp に送信し,\ecp の標準出力に出力する.送信する文字列は改行までであり,入力の終わりはEOFを読み込んだ時である.
\item 接続が切れた時,\esp 側で"closed"と出力する.
\item \esp は\ecp との接続が一旦終了しても次に接続されるまで起動し続ける.
\item 入力できる文字列の長さは半角で1023文字である.
\end{ite}

\subsection{実行結果}
\label{sec:ssh}

以下に実行結果を記載する.

\begin{bit}[l]{sshによるリモートログイン}
\begin{verbatim}
 1	r-kawakm@exp048:~$ ssh exp068
 2	The authenticity of host 'exp068 (192.168.16.47)' can't be established.
 3	ECDSA key fingerprint is
    SHA256:7b17ZyM1wvrsUa5xGcYXoFe18QYkzcrS44oSk02MOAg.
 4	Are you sure you want to continue connecting (yes/no)? yes
 5	Warning: Permanently added 'exp068,192.168.16.47' (ECDSA) to the list of
    known hosts.
 6	r-kawakm@exp068's password:
 7	Welcome to Ubuntu 16.04.4 LTS (GNU/Linux 4.4.0-124-generic x86_64)
 8
 9	 * Documentation:  https://help.ubuntu.com
10	 * Management:     https://landscape.canonical.com
11	 * Support:        https://ubuntu.com/advantage
12
13	0 個のパッケージがアップデート可能です。
14	0 個のアップデートはセキュリティアップデートです。
\end{verbatim}
\end{bit}

ターミナルのウィンドウを2つ開いて,片方のウィンドウでsshコマンドを用いて,exp068というホストにリモートログインした.2つのターミナルのウィンドウが別々のホストに接続されていることによって,\esp と\ecp が端末内でメッセージのやり取りをしているのではなく,ネットワークを通じてメッセージをやり取りしているとわかる.表\ref{tab:host_name}に\esp と\ecp のホストの割り当てについてまとめてある.

\begin{tab}[htb]
\centering
\begin{tabu}{|c|c|c|}
\hline
プログラム & \esp & \ecp \\
\hline
 ホスト名 & exp048 & exp068 \\
\hline
\end{tabu}
\caption{(課題2-1)における \esp と\ecp のホスト名}
\label{tab:host_name}
\end{tab}

\begin{bit}[l]{\esp 実行結果}
\begin{verbatim}
 1	r-kawakm@exp048:~/enshuC/enshuC_kadai2$ ./echoserver.exe
 2	[exp068.exp.ics.es.osaka-u.ac.jp]
 3  closed
\end{verbatim}
\end{bit}

\esp を実行しておき,2行目でホストexp068との接続が確立されたことがわかった.3行目では通信相手である\ecp が通信を終了したことが"closed"と出力されていることによって確認できた.

\begin{bit}[l]{\ecp 実行結果}
\begin{verbatim}
 1	r-kawakm@exp068:~/enshuC/enshuC_kadai2$ ./echoclient.exe exp048
 2	echoclient->
 3	echoclient->
 4  Hello World!
 5  Hello World!
\end{verbatim}
\end{bit}

ホストexp048で\esp を実行している状態で,先ほどsshコマンドによってホストexp068にリモートログインしたターミナルのウィンドウ上で\ecp を実行し,\esp 側 のウィンドウで"[exp068.exp.ics.es.osaka-u.ac.jp]"という出力があることで,接続されたことが確認できる.配布された手引きでは"echclient host"で実行するよう指示されているが,コンパイル時に生成ファイル名をechoclientに設定すればそのようになり,ファイル名の違いなので問題ないと考えている.

2,4行目の文字列が入力した文字列であり,3,5行目でおうむ返しのように入力した文字列と全く同じ文字列が出力されていることが確認でき,"Ctrl+D"で\ecp が終了したため,\ref{sec:echo_shiyo}節で記載した仕様(文字列の長さ以外)を満たしていることが確認できた.文字列の長さについての仕様の確認は行っていないが,文字列を保存する配列の長さが1024であり,文字列の終端記号の分を除いた1023個を入力文字列に当てることができることから仕様が満たされていることを確認できる.

空白を含んだ文字列でも問題なく処理できていることに注意して,\ref{sec:echo_explanation}節に移ってほしい.


\subsection{プログラム説明}
\label{sec:echo_explanation}

この節では\ecp の説明を行う.\esp は配布された資料のものを実行しているため,本レポートでは説明しない.はじめにも伝えたが,\esp と\ecp の全行のコードは\ref{sec:echoserver_code},\ref{sec:echoclient_code}節に記載している.

\begin{bit}[l]{ポート番号の定義}
\begin{verbatim}
10	#define PORT 10120
\end{verbatim}
\end{bit}

10行目でポート番号を定義していて,\esp と\ecp は10120番を使用する.

\begin{bit}[l]{変数の宣言}
\begin{verbatim}
13	  int sock;
14	  char str[1024], receive[1024];
15	  struct sockaddr_in host;
16	  struct hostent *hp;
\end{verbatim}
\end{bit}

13-16行で本プログラムで使用する変数を宣言している.変数の型と名前,用途を表\ref{tab:echoclient_variable}にまとめる.

\begin{tab}[htb]
\centering
\begin{tabu}{|l|l||l|}
\hline
型 & 変数名 & 用途 \\
\hline
\hline
 int型 & sock & ソケットの生成に用いる \\
\hline
 char型 & str[1024] & 入力された文字列を保存する \\
\hline
 char型 & receive[1024] & \esp から受信した文字列を保存する \\
\hline
 構造体 sockaddr\_in型 & host & 接続先のソケットアドレスを保存する \\
\hline
 構造体 hostent型 & *hp & \esp のホスト情報を保存する \\
\hline
\end{tabu}
\caption{\ecp で使用する変数}
\label{tab:echoclient_variable}
\end{tab}

\begin{bit}[l]{引数の数の確認}
\begin{verbatim}
18	  /* コマンド引数が1つであることを確認 */
19	  if(argc != 2) {
20	    fprintf(stderr, "usage: %s host\n", argv[0]);
21	    exit(1);
22	  }
\end{verbatim}
\end{bit}

コマンド引数が2つであることを確認し,コマンド引数が2つでない時は間違った実行方法であるため,正しい実行方法をエラー出力して,\ecp を終了する.

\begin{bit}[l]{ソケットの生成}
\begin{verbatim}
24	  /* ソケットの生成 */
25	  if((sock = socket(AF_INET, SOCK_STREAM, 0)) < 0) {
26	    perror("client: socket");
27	    exit(1);
28	  }
\end{verbatim}
\end{bit}

socket関数を用いることで通信の出入り口となるソケットを生成する.また,socket関数の返り値が0以下であれば,正しく生成できていないためエラー出力をして,\ecp を終了する.

\begin{bit}[l]{接続先ホストの設定,ホスト情報の保存}
\begin{verbatim}
30	  /*
31	   * host構造体に接続するサーバーのアドレス・ポート番号を設定
32	   */
33
34	  bzero((char *)&host, sizeof(host));
35	  host.sin_family = AF_INET;
36	  host.sin_port = htons(PORT);
37
38	  if((hp = gethostbyname(argv[1])) == NULL) {
39	    fprintf(stderr, "unkown host %s\n", argv[1]);
40	    close(sock);
41	    exit(1);
42	  }
43	  bcopy(hp->h_addr, &host.sin_addr, hp->h_length);
\end{verbatim}
\end{bit}

接続先ホストの設定を変数hostに反映させて,接続する準備を行う.設定を行う前にbzero関数を用いて変数の初期化を忘れないようにすること.

その後,実行時に引数として与えられたホスト名のホストに接続を試み,成功すればホスト情報をポインタ変数*hpにbcopy関数を用いて保存し,失敗すればエラー出力をしてソケットを閉じた後,\ecp を終了する.

\begin{bit}[l]{ソケットの接続}
\begin{verbatim}
45	  /* ソケットをサーバに接続 */
46	  if (connect(sock, (struct sockaddr *)&host, sizeof(host)) > 0) {
47	    perror("client: connect");
48	    close(sock);
49	    exit(1);
50	  }
\end{verbatim}
\end{bit}

connect関数を用いて生成したソケットを\esp のソケットと接続する.接続に失敗した時にはエラー出力をしてソケットを閉じた後,\ecp を終了する.

\begin{bit}[l]{入力,送受信,出力}
\begin{verbatim}
52	memset(str, '\0', sizeof(str));
53
54	while (read(0, str, sizeof(str)) > 0) {
55	  memset(receive, '\0', sizeof(receive));
56	  if(write(sock, str, sizeof(str)) < 0) {
57	    perror("write");
58	    exit(1);
59	  }
60
61	  while (read(sock, receive, sizeof(receive)) < 0) {
62	    perror("read");
63	  }
64	    write(1, receive, sizeof(receive));
65	  // printf("%s\n", receive);
66	}
67
68	  /* ソケットをクローズ */
69	  close(sock);
70	}
\end{verbatim}
\end{bit}

はじめに入力文字列を保存する変数str,receiveをmemset関数を用いることで初期化する.初期化することで,文字化けのようなゴミ文字列を出力しないようにする.

まず,read関数で標準入力から文字列を受け取り,変数strに格納する.その後write関数を用いてソケットを介したデータ送出を行う.送信した後,サーバーからデータを受信するまでread関数を用いて,受け取った後write関数を用いて\ecp 側の標準出力に受け取った文字列を出力する.そして,次の入力を受け付ける.この処理をEOF(End Of File)を受け取るまで行い,EOFを受け取ったら,ソケットを閉じれ\ecp の終了となる.

はじめ,標準入力からの文字列の受け取りにscanf関数を用いて,出力にprintf関数を用いていたが,この時は空白を含んだ文字列が正しく処理することができず,1行で出力されないといけないところが,下の例のように空白の部分で改行されて文字列が出力されていた.なぜ,そのような結果になったか解明できていないが,そのような結果になることを避けるためにwrite関数とread関数を用いてプログラムを記述した.

\begin{verbatim}
例
入力 : Hello World!
出力 : Hello
       World!
\end{verbatim}

\section{(課題2-2) simple-talkプログラム}

この章では簡易talkプログラムについて記述する.簡易talkプログラムは\esp と\ecp のように\ecp が送信した内容がおうむ返しで\ecp に帰ってくると言うブーメランのような機能ではなくて,\sts と\stc が各々入力した文字列が相手のウィンドウにリアルタイムで出力されるというチャットのような機能を有しているプログラムである.

また,作成した\sts と\stc のコードを\ref{sec:simple-talk-server_code},\ref{sec:simple_talk_client_code}節に記載している.

\subsection{課題内容}

本課題の配布資料である手引きを参考に\sts と\stc を作成せよ.

\subsection{仕様}

\sts と\stc の仕様は以下の通りである.

\begin{ite}
\item 実行のコマンドは"echoclient host"と入力する.hostの部分には\sts を実行しているホスト名を指定すること.
\item \sts と\stc はポート番号10130を介して接続する.
\item \sts の標準入力に入力された文字列はリアルタイムで\stc に送信されて\stc の標準出力に出力される.逆に\stc の標準入力に入力された文字列も同様にリアルタイムで\sts に送信されて\sts の標準出力に出力される.
\item \sts と\stc の接続が確立した時,両プログラムの標準出力に"connected"と出力することによって,両プログラムのユーザーに接続の確立を伝える.
\item \sts と\stc は"Ctrl + D"でEOF(End oF File)がどちらかに入力された時に接続を終了する.
\item 接続が切れた時,\sts と\stc の両方で"closed"と標準出力に出力する.
\item \sts は\stc との接続が一旦終了しても次の接続されるまで起動し続ける.
\item 入力できる文字列の長さは半角で1023文字である.
\end{ite}

\subsection{実行結果}
\label{sec:talk_result}

sshコマンドを用いたリモートログインの実行結果は\ref{sec:ssh}節の"sshによるリモートログイン"と同じため,省略する.

\begin{bit}[l]{simple\_talk\_server}
\begin{verbatim}
 1	r-kawakm@exp048:~/enshuC$ ./simple_talk_server.exe
 2	[exp068.exp.ics.es.osaka-u.ac.jp]
 3	connected
 4	cl->sv
 5	sv->cl
 6	closed
 7
\end{verbatim}
\end{bit}

配布された手引きでは"simple-talk-server"で実行するよう指示されているが,コンパイル時に生成ファイル名をに"simple-talk-server"設定すればそのようになり,ファイル名の違いなので問題ないと考えている.

2,3行目がホストexp068の\stc との接続が確立したことをユーザーに知らせる出力である."cl->sv"は\stc から送信されたデータを\sts で受け取り,出力している例であり,"sv->cl"は\sts から送信されたデータを\stc で受け取り,出力している例である.ここでは\stc から送られてきた情報を受信して,\sts で出力していることを確認しておく.

6行目では"closed"と出力されていることから通信が終了したことは確認できる.その後次の接続の待機に移行している.

\begin{bit}[l]{simple\_talk\_client}
\begin{verbatim}
  1	r-kawakm@exp068:~/enshuC$ ./simple_talk_client.exe exp048
  2	connected
  3	cl->sv
  4	sv->cl
  5	closed
\end{verbatim}
\end{bit}

1行目で仕様の通りにホスト名を引数として与えて実行している.配布された手引きでは"simple-talk-client host"で実行するよう指示されているが,コンパイル時に生成ファイル名をに"simple-talk-client"設定すればそのようになり,ファイル名の違いなので問題ないと考えている.また2行目において"connected"と出力されていることから\sts との接続が確立したことが確認できる.3,4行目については\sts の出力結果のところで記した通りであり,ここでは\sts から送られてきた情報を受信して,\stc で出力していることを確認しておく.6行目では"closed"と出力されていることから通信が終了したことは確認できる.接続の終了は\sts ,\stc のどちらの標準入力でEOFが入力されたときに行われる.

文字列の長さについての仕様の確認は行っていないが,文字列を保存する配列の長さが1024であり,文字列の終端記号の分を除いた1023個を入力文字列に当てることができることから仕様が満たされていることを確認できる.

\sts と\stc の出力結果から送受信がうまく行われていることが確認できた.しかし,次のような場合には少し変な挙動を起こす.ただし,今回の課題で求められていることは満たしているため問題はないと考えている.

\begin{bit}[l]{入力中にデータを受信した時:受信側}
\begin{verbatim}
 1	r-kawakm@exp048:~/enshuC$ ./simple_talk_server.exe
 2	[exp068.exp.ics.es.osaka-u.ac.jp]
 3	connected
 4	sv->cl->sv
 5
\end{verbatim}
\end{bit}

\begin{bit}[l]{入力中にデータを受信した時:送信側}
\begin{verbatim}
  1	r-kawakm@exp068:~/enshuC$ ./simple_talk_client.exe exp048
  2	connected
  3	cl->sv
\end{verbatim}
\end{bit}

この出力の際の状況としては,\sts の標準入力に"sv->"と入力している途中で,\stc の標準入力に"cl->sv"と入力して送信したところ,\sts への出力が入力途中であった"sv->"に続いて"sv->cl->sv"となっているのである.これは\stc が入力中で\sts が送信してきても同じである.入力の途中にデータを受信した際に改行して出力することも考えたが,そうすると入力を続けることが困難になると考え,さらに課題の最低限の仕様を満たしていると考えたためにこのままにしている.

LINEアプリのように入力フィールドと出力フィールドが分かれていれば良いが,本課題のように1つのターミナルウィンドウで入出力を行う場合には改善は難しいのではないかと考えている.対処方法としては入力を受け付けている時にデータを受信しても,入力が終わるまで出力しないというのが考えられるが,これではリアルタイムとは言い難いのではないかと考えている.

\subsection{プログラム説明}

この節では\sts と\stc の説明を行う.

\subsubsection{\sts の説明}

まず,\sts について説明する.

\begin{bit}[l]{ポート番号の定義}
\begin{verbatim}
 9	#define PORT 10130
\end{verbatim}
\end{bit}

9行目でポート番号を定義している.\sts と\stc は10130番を使用する.

\begin{bit}[l]{変数の宣言}
\begin{verbatim}
12	  int sock, csock; /* クライアントを受け付けたソケット*/
13	  struct sockaddr_in svr;
14	  struct sockaddr_in clt;
15	  struct hostent *cp;
16	  int clen;
17	  char sbuf[1024], cbuf[1024];
18	  int nbytes;
19	  int reuse;
20	  fd_set rfds; /* select() で用いるファイル記述子集合*/
21	  struct timeval tv; /* select() が返ってくるまでの待ち時間を指定する変数*/
\end{verbatim}
\end{bit}

以下の表\ref{tab:simpletalkserver_variable}に\sts で使用する変数の型と名前,用途をまとめる.

\begin{tab}[htb]
\centering
\begin{tabu}{|l|l||l|}
\hline
型 & 変数名 & 用途 \\
\hline
\hline
 int型 & sock & 送信用ソケットの生成に用いる \\
\hline
 int型 & csock & 受信用ソケットの生成に用いる \\
\hline
構造体 sockaddr\_in型 & svr & 受付用のソケットアドレスを保存する \\
\hline
構造体 sockaddr\_in型 & clt & 接続先のソケットアドレスを保存する \\
\hline
構造体 hostent型 & *cp & \stc のホスト情報を保存する \\
\hline
 int型 & clen & 変数cltのサイズを保存する \\
\hline
 char型 & sbuf[1024] & 入力された文字列を保存する \\
\hline
 char型 & cbuf[1024] & \stc から受信した文字列を保存する \\
\hline
 int型 & nbytes & read関数の返り値を保存する. \\
\hline
 int型 & reuse & ソケットの再利用に使用する \\
\hline
 fd\_set型 & rfds & select関数で用いるファイル記述子の集合 \\
\hline
 構造体 timeval型 & tv & select関数が帰ってくるまでの待ち時間を指定する \\
\hline
\end{tabu}
\caption{\sts で使用する変数}
\label{tab:simpletalkserver_variable}
\end{tab}

\begin{bit}[l]{ソケットの生成}
\begin{verbatim}
23	  /* ソケットの生成 */
24	  if ((sock = socket(AF_INET, SOCK_STREAM, IPPROTO_TCP)) < 0) {
25	    perror("socket");
26	    exit(1);
27	  }
\end{verbatim}
\end{bit}

socket関数を用いることで通信の出入り口となるソケットを生成する.また,socket関数の返り値が0以下であれば,正しく生成できていないためエラー出力をして,\sts を終了する.

\begin{bit}[l]{ソケットアドレス再利用の指定}
\begin{verbatim}
29	  /* ソケットアドレス再利用の指定 */
30	  reuse = 1;
31	  if(setsockopt(sock, SOL_SOCKET, SO_REUSEADDR, &reuse, sizeof(reuse))
    < 0) {
32	    perror("setsockopt");
33	    exit(1);
34	  }
\end{verbatim}
\end{bit}

ここではソケットアドレスの再利用をするための処理を行っている.

\begin{bit}[l]{クライアントを受け付けるためのソケット情報の設定}
\begin{verbatim}
36	  /* client 受付用ソケットの情報設定 */
37	  bzero(&svr, sizeof(svr));
38	  svr.sin_family = AF_INET;
39	  svr.sin_addr.s_addr = htonl(INADDR_ANY);
40
41	  /* 受付側の IP アドレスは任意 */
42	  svr.sin_port = htons(PORT); /* ポート番号 10130 を介して受け付ける */
43
44	  /* ソケットにソケットアドレスを割り当てる */
45	  if(bind(sock, (struct sockaddr *)&svr, sizeof(svr)) < 0) {
46	    perror("bind");
47	    exit(1);
48	  }
\end{verbatim}
\end{bit}

\stc を受け付けるためのソケットの情報を設定している.設定する前のbzero関数を用いて変数をsvrを初期化している.設定した後,ポート番号を10130番に設定している.そして,bind関数を用いて変数sockと変数svrを結びつけている.この時結びつけに失敗した時にはエラー出力をして\sts を終了する.

\begin{bit}[l]{待ち受けクライアント数の設定}
\begin{verbatim}
50	  /* 待ち受けクライアント数の設定 */
51	  if(listen(sock, 5) < 0) { /* 待ち受け数に 5 を指定 */
52	    perror("listen");
53	    exit(1);
54	  }
\end{verbatim}
\end{bit}

listen関数を用いてクライアントの受付数を5に設定している.指定できなかった時にはエラー出力をして\sts を終了する.

\begin{bit}[l]{受付,入力,受信,出力,終了}
\begin{verbatim}
55	  do {
56
57	    /* クライアントの受付 */
58	    clen = sizeof(clt);
59	    if ((csock = accept(sock, (struct sockaddr *)&clt, &clen)) < 0) {
60	      perror("accept");
61	      exit(2);
62	    }
63
64	    /* クライアントのホスト情報の取得 */
65	    cp = gethostbyaddr((char *)&clt.sin_addr, sizeof(struct in_addr),
        AF_INET);
66	    printf("[%s]\n", cp->h_name);
67	    printf("connected\n");
68
69	    nbytes = -1;
70
71	    do{
72	      /* 入力を監視するファイル記述子の集合を変数rfds にセットする*/
73	      FD_ZERO(&rfds); /* rfds を空集合に初期化*/
74	      FD_SET(0, &rfds); /* 標準入力*/
75	      FD_SET(csock, &rfds); /* クライアントを受け付けたソケット*/
76	      /* 監視する待ち時間を1 秒に設定*/
77	      tv.tv_sec = 1;
78	      tv.tv_usec = 0;
79	      memset(sbuf, '\0', sizeof(sbuf));
80	      memset(cbuf, '\0', sizeof(cbuf));
81
82	      /* 標準入力とソケットからの受信を同時に監視する */
83	      if(select(csock+1, &rfds, NULL, NULL, &tv) > 0) {
84	        if(FD_ISSET(0, &rfds)) { /* 標準入力から入力があったなら*/
85	          if((nbytes = read(0, sbuf, sizeof(sbuf))) > 0){
              /*標準入力から読み込みクライアントに送信 */
86	            write(csock, sbuf, nbytes);
87	          }
88	        }
89	        if(FD_ISSET(csock, &rfds)) { /* ソケットから受信したなら*/
90	        /* ソケットから読み込み端末に出力*/
91	         if((nbytes = read(csock, cbuf, sizeof(cbuf))) > 0){
92	            write(1, cbuf, nbytes);
93	         }
94	        }
95	      }
96	    } while (nbytes != 0);
97
98	    /* read() が 0 を返すまで (=End-Of-File) 繰り返す */
99	    close(csock);
100	    printf("closed\n");
101	  } while(1); /* 繰り返す*/
\end{verbatim}
\end{bit}

57-67行目ではクライアントを受け付けて,クライアントのホスト情報を取得する.73-78行目では入力を監視するファイル記述子の集合を変数rfdsに保存する.73行目で変数rfdsをFD\_ZEROマクロを用いて初期化する.74,75行目でFD\_SETマクロの第1引数に0,csockを与えることで標準入力と\stc からの受信を代入している.また,77-78行目で入力を監視する待ち時間を1秒に設定している.79-80行目では標準入力とソケットから受信した文字列を保存する変数sbuf,cbufの初期化をmemset関数を用いて行っている.83-96行目ではselect関数を用いて標準入力とソケットからの受信を同時に監視している.標準入力から入力があった時にはread関数で変数sbufにその入力を保存して,write関数で\stc に送信する.ソケットから受信したなら受信した文字列を変数cbufにその文字列を保存して,write関数で標準出力にその文字列を出力する.この監視の処理をEOF(End Of File)を受け付けるまで繰り返す.EOFを受け付けたら接続されている\stc のソケットを閉じて,接続が終了したことを"closed"と標準出力に出力することでユーザーに知らせる.そして他のクライアントプログラムからのソケット受信を待ち続ける.

\subsubsection{\stc の説明}

ここでは\stc の説明を行う.

\begin{bit}[l]{ポート番号の定義}
\begin{verbatim}
 9	#define PORT 10130
\end{verbatim}
\end{bit}

9行目でポート番号を定義している.\sts と\stc は10130番を使用する.

\begin{bit}[l]{変数の宣言}
\begin{verbatim}
12		int sock, nbytes;
13		char sbuf[1024], *name, cbuf[1024];
14		struct sockaddr_in svr;
15		struct hostent *hp;
16		fd_set rfds;
17		struct timeval tv;
\end{verbatim}
\end{bit}

以下の表\ref{tab:simpletalkclient_variable}に\stc で使用する変数の型と名前,用途をまとめる.

\begin{tab}[H]
\centering
\begin{tabu}{|l|l||l|}
\hline
型 & 変数名 & 用途 \\
\hline
\hline
 int型 & sock & 送信用ソケットの生成に用いる \\
\hline
int型 & nbytes & read関数の返り値を保存する. \\
\hline
char型 & *name &  \\
\hline
char型 & sbuf[1024] & \sts から受信した文字列を保存する \\
\hline
char型 & cbuf[1024] & 入力された文字列を保存する \\
\hline
構造体 sockaddr\_in型 & svr & 受付用のソケットアドレスを保存する \\
\hline
構造体 hostent型 & *hp & \sts のホスト情報を保存する \\
\hline
 fd\_set型 & rfds & select関数で用いるファイル記述子の集合 \\
\hline
 構造体 timeval型 & tv & select関数が帰ってくるまでの待ち時間を指定する \\
\hline
\end{tabu}
\caption{\stc で使用する変数}
\label{tab:simpletalkclient_variable}
\end{tab}

\begin{bit}[l]{引数の数の確認}
\begin{verbatim}
19		if(argc != 2) {
20			fprintf(stderr, "Usage: %s hostname \n", argv[0]);
21			exit(1);
22		}else{
23		  name = *(argv+1);
24		}
\end{verbatim}
\end{bit}

コマンド引数が2つであることを確認し,コマンド引数が2つでない時は間違った実行方法であるため,正しい実行方法をエラー出力して,\stc を終了する.

\begin{bit}[l]{ソケットの生成}
\begin{verbatim}
26		if((sock = socket(AF_INET, SOCK_STREAM, IPPROTO_TCP)) < 0){
27		  perror("socket");
28		  exit(1);
29		}
\end{verbatim}
\end{bit}

socket関数を用いることで通信の出入り口となるソケットを生成する.また,socket関数の返り値が0以下であれば,正しく生成できていないためエラー出力をして,\stc を終了する.

\begin{bit}[l]{接続先ホストの設定,ホスト情報の保存}
\begin{verbatim}
31		bzero(&svr, sizeof(svr));
32		svr.sin_family = AF_INET;
33		svr.sin_port = htons(PORT);
34		if((hp = gethostbyname(name)) == NULL){
35		  perror("gethostbyname");
36		  exit(1);
37		}
38		bcopy(hp->h_addr, &svr.sin_addr, hp->h_length);
\end{verbatim}
\end{bit}

接続先ホストの設定を変数hostに反映させて,接続する準備を行う.設定を行う前にbzero関数を用いて変数の初期化を忘れないようにすること.

その後,実行時に引数として与えられたホスト名のホストに接続を試み,成功すればホスト情報をポインタ変数*hpにbcopy関数を用いて保存し,失敗すればエラー出力をしてソケットを閉じた後,\stc を終了する.

\begin{bit}[l]{ソケットの接続}
\begin{verbatim}
40		if((connect(sock, (struct sockaddr *)&svr, sizeof(svr))) < 0){
41		    perror("connect");
42		    exit(1);
43		}
44		else {
45			printf("connected\n" );
46		}
\end{verbatim}
\end{bit}

connect関数を用いて生成したソケットを\stc のソケットと接続する.接続に失敗した時にはエラー出力をしてソケットを閉じた後,\stc を終了する.接続に成功した時はユーザーに接続が確立したことを伝えるために標準出力に"connected"と出力する.

\begin{bit}[l]{入力,受信,出力,終了}
\begin{verbatim}
48		do{
49		  FD_ZERO(&rfds);
50		  FD_SET(0, &rfds);
51		  FD_SET(sock, &rfds);
52		  tv.tv_sec = 1;
53		  tv.tv_usec = 0;
54		  memset(sbuf, '\0', sizeof(sbuf));
55			memset(cbuf, '\0', sizeof(cbuf));
56
57		  if(select(sock+1, &rfds, NULL, NULL, &tv) > 0){
58			  if(FD_ISSET(sock, &rfds)){
59			    if((nbytes = read(sock, cbuf, sizeof(cbuf))) > 0){
60			      write(1, cbuf, nbytes);
61			    }
62			  }
63			  if(FD_ISSET(0, &rfds)){
64			    if((nbytes = read(0, sbuf, sizeof(sbuf))) > 0){
65			      write(sock, sbuf, nbytes);
66			    }
67			  }
68		  }
69		} while(nbytes != 0);
\end{verbatim}
\end{bit}

49-53行目では入力を監視するファイル記述子の集合を変数rfdsに保存する.49行目で変数rfdsをFD\_ZEROマクロを用いて初期化する.50,51行目でFD\_SETマクロの第1引数に0,sockを与えることで標準入力と\sts からの受信を代入している.また,52-53行目で入力を監視する待ち時間を1秒に設定している.54-55行目では標準入力とソケットから受信した文字列を保存する変数sbuf,cbufの初期化をmemset関数を用いて行っている.57-68行目ではselect関数を用いて標準入力とソケットからの受信を同時に監視している.標準入力から入力があった時にはread関数で変数sbufにその入力を保存して,write関数で\stc に送信する.ソケットから受信したなら受信した文字列を変数cbufにその文字列を保存して,write関数で標準出力にその文字列を出力する.この監視の処理をEOF(End Of File)を受け付けるまで繰り返す.

\begin{bit}[l]{終了処理}
\begin{verbatim}
70		close(sock);
71		printf("closed\n");
\end{verbatim}
\end{bit}

EOFを受け付けたら接続されている\sts のソケットを閉じて,接続が終了したことを"closed"と標準出力に出力することでユーザーに知らせる.そして\stc を終了する.

\section{考察}

本課題で作成した簡易talkプログラムには\ref{sec:talk_result}節でも記したような不便な部分がある.入力の途中でソケットからデータを受信すると入力している文字列に続いて受け取ったデータを出力するというのは少し問題であるが,該当箇所でも記したように,1つのターミナルウィンドウで入出力を行っていて,かつリアルタイムで反映させるならば仕方がないことであったと考える.

他にはLINEアプリなどのSNSの機能を考えると複数のユーザーとのトークが可能になるような機能を実装できると考えると本当に簡易的なものだと思える.時間があれば,発展問題にも取り組んでさらなる拡張を行いたかったが,残念である.

\section{感想}

情報ネットワークの講義で初めてネットワークについての勉強を行った状態,つまりネットワークに関する知識が全くない状態で本課題にのぞんだため,とっかかりを探すのに大変困った.ネットワークがそのように接続されて,どのように通信を行っているかも知らなかったため,ソケットについても全くわからずいろいろ調べながら理解して行った.そこそこの知識が揃うと配布された手引きのわかりやすさに感動して,課題を進めることができた.プログラムを組み終えた今さらなるネットワークの知識を得ることを楽しみにしているため,このモチベーションのまま本講義の終了まで突っ走っていきたいと考えている.

\appendix % appendixは、「これ以降は付録」ということを表す

\section{(付録) プログラムコード}

\subsection{(課題2-1) echoserver.c}
\label{sec:echoserver_code}
\small{
\begin{verbatim}
 1  #include <sys/types.h>
 2  #include <sys/socket.h>
 3  #include <netinet/in.h>
 4  #include <netdb.h>
 5  #include <unistd.h>
 6  #include <string.h>
 7  #include <stdio.h>
 8  #include <stdlib.h>
 9
10  int main(int argc,char **argv) {
11    int sock,csock;
12    struct sockaddr_in svr;
13    struct sockaddr_in clt;
14    struct hostent *cp;
15    int clen;
16    char rbuf[1024];
17    int nbytes;
18    int reuse;
19    /* ソケットの生成*/
20    if ((sock=socket(AF_INET,SOCK_STREAM,IPPROTO_TCP))<0) {
21      perror("socket");
22      exit(1);
23    }
24    /* ソケットアドレス再利用の指定*/
25    reuse=1;
26    if(setsockopt(sock,SOL_SOCKET,SO_REUSEADDR,&reuse,sizeof(reuse))<0) {
27      perror("setsockopt");
28      exit(1);
29    }
30    /* client 受付用ソケットの情報設定*/
31    bzero(&svr,sizeof(svr));
32    svr.sin_family=AF_INET;
33    svr.sin_addr.s_addr=htonl(INADDR_ANY);
34    /* 受付側のIP アドレスは任意*/
35    svr.sin_port=htons(10120); /* ポート番号10120 を介して受け付ける*/
36    /* ソケットにソケットアドレスを割り当てる*/
37    if(bind(sock,(struct sockaddr *)&svr,sizeof(svr))<0) {
38      perror("bind");
39      exit(1);
40    }
41    /* 待ち受けクライアント数の設定*/
42    if (listen(sock,5)<0) { /* 待ち受け数に5 を指定*/
43      perror("listen");
44      exit(1);
45    }
46    do {
47    /* クライアントの受付*/
48      clen = sizeof(clt);
49      if ( ( csock = accept(sock,(struct sockaddr *)&clt,&clen) ) <0 ) {
50        perror("accept");
51        exit(2);
52      }
53    /* クライアントのホスト情報の取得*/
54      cp = gethostbyaddr((char *)&clt.sin_addr,sizeof(struct in_addr),AF_INET);
55      printf("[%s]\n",cp->h_name);
56      do {
57        memset(rbuf, '\0', sizeof(rbuf));
58        /* クライアントからのメッセージ受信*/
59        if ( ( nbytes = read(csock,rbuf,sizeof(rbuf)) ) < 0) {
60          perror("read");
61        } else {
62          write(csock,rbuf,nbytes);
63          /* 受信文字列をそのままクライアントへ返す(echo) */
64        }
65      } while ( nbytes != 0 );
66      /* read() が0 を返すまで(=End-Of-File) 繰り返す*/
67      close(csock);
68      printf("closed\n");
69    }
70    while(1); /* 次の接続要求を繰り返し受け付ける*/
71  }
\end{verbatim}
}

\subsection{(課題2-1) echoclient.c}
\label{sec:echoclient_code}
\small{
\begin{verbatim}
 1  #include <sys/types.h> /* socket() を使うために必要*/
 2  #include <sys/socket.h> /* 同上 */
 3  #include <netinet/in.h> /* INET ドメインのソケットを使うために必要*/
 4  #include <arpa/inet.h> /* 同上 */
 5  #include <netdb.h>
 6  #include <stdio.h>
 7  #include <string.h>
 8  #include <stdlib.h>
 9  #include <unistd.h>
10  #define PORT 10120
11
12  int main(int argc, char **argv) {
13    int sock;
14    char str[1024], receive[1024];
15    struct sockaddr_in host;
16    struct hostent *hp;
17
18    /* コマンド引数が1つであることを確認 */
19    if(argc != 2) {
20      fprintf(stderr, "usage: %s host\n", argv[0]);
21      exit(1);
22    }
23
24    /* ソケットの生成 */
25    if((sock = socket(AF_INET, SOCK_STREAM, 0)) < 0) {
26      perror("client: socket");
27      exit(1);
28    }
29
30    /*
31     * host構造体に接続するサーバーのアドレス・ポート番号を設定
32     */
33
34    bzero((char *)&host, sizeof(host));
35    host.sin_family = AF_INET;
36    host.sin_port = htons(PORT);
37
38    if((hp = gethostbyname(argv[1])) == NULL) {
39      fprintf(stderr, "unkown host %s\n", argv[1]);
40      close(sock);
41      exit(1);
42    }
43    bcopy(hp->h_addr, &host.sin_addr, hp->h_length);
44
45    /* ソケットをサーバに接続 */
46    if (connect(sock, (struct sockaddr *)&host, sizeof(host)) > 0) {
47      perror("client: connect");
48      close(sock);
49      exit(1);
50    }
51
52  memset(str, '\0', sizeof(str));
53
54  while (read(0, str, sizeof(str)) > 0) {
55    memset(receive, '\0', sizeof(receive));
56    if(write(sock, str, sizeof(str)) < 0) {
57      perror("write");
58      exit(1);
59    }
60
61    while (read(sock, receive, sizeof(receive)) < 0) {
62      perror("read");
63    }
64      write(1, receive, sizeof(receive));
65    // printf("%s\n", receive);
66  }
67
68    /* ソケットをクローズ */
69    close(sock);
70  }
\end{verbatim}
}

\subsection{(課題2-2) simple\_talk\_server.c}
\label{sec:simple-talk-server_code}
\small{
\begin{verbatim}
 1  #include <sys/types.h>
 2  #include <sys/socket.h>
 3  #include <netinet/in.h>
 4  #include <netdb.h>
 5  #include <unistd.h>
 6  #include <string.h>
 7  #include <stdio.h>
 8  #include <stdlib.h>
 9  #define PORT 10130
10
11  int main(int argc, char **argv) {
12    int sock, csock; /* クライアントを受け付けたソケット*/
13    struct sockaddr_in svr;
14    struct sockaddr_in clt;
15    struct hostent *cp;
16    int clen;
17    char sbuf[1024], cbuf[1024];
18    int nbytes;
19    int reuse;
20    fd_set rfds; /* select() で用いるファイル記述子集合*/
21    struct timeval tv; /* select() が返ってくるまでの待ち時間を指定する変数*/
22
23    /* ソケットの生成 */
24    if ((sock = socket(AF_INET, SOCK_STREAM, IPPROTO_TCP)) < 0) {
25      perror("socket");
26      exit(1);
27    }
28
29    /* ソケットアドレス再利用の指定 */
30    reuse = 1;
31    if(setsockopt(sock, SOL_SOCKET, SO_REUSEADDR, &reuse, sizeof(reuse)) < 0) {
32      perror("setsockopt");
33      exit(1);
34    }
35
36    /* client 受付用ソケットの情報設定 */
37    bzero(&svr, sizeof(svr));
38    svr.sin_family = AF_INET;
39    svr.sin_addr.s_addr = htonl(INADDR_ANY);
40
41    /* 受付側の IP アドレスは任意 */
42    svr.sin_port = htons(PORT); /* ポート番号 10130 を介して受け付ける */
43
44    /* ソケットにソケットアドレスを割り当てる */
45    if(bind(sock, (struct sockaddr *)&svr, sizeof(svr)) < 0) {
46      perror("bind");
47      exit(1);
48    }
49
50    /* 待ち受けクライアント数の設定 */
51    if(listen(sock, 5) < 0) { /* 待ち受け数に 5 を指定 */
52      perror("listen");
53      exit(1);
54    }
55    do {
56
57      /* クライアントの受付 */
58      clen = sizeof(clt);
59      if ((csock = accept(sock, (struct sockaddr *)&clt, &clen)) < 0) {
60        perror("accept");
61        exit(2);
62      }
63
64      /* クライアントのホスト情報の取得 */
65      cp = gethostbyaddr((char *)&clt.sin_addr, sizeof(struct in_addr), AF_INET);
66      printf("[%s]\n", cp->h_name);
67      printf("connected\n");
68
69      nbytes = -1;
70
71      do{
72        /* 入力を監視するファイル記述子の集合を変数rfds にセットする*/
73        FD_ZERO(&rfds); /* rfds を空集合に初期化*/
74        FD_SET(0, &rfds); /* 標準入力*/
75        FD_SET(csock, &rfds); /* クライアントを受け付けたソケット*/
76        /* 監視する待ち時間を1 秒に設定*/
77        tv.tv_sec = 1;
78        tv.tv_usec = 0;
79        memset(sbuf, '\0', sizeof(sbuf));
80        memset(cbuf, '\0', sizeof(cbuf));
81
82        /* 標準入力とソケットからの受信を同時に監視する */
83        if(select(csock+1, &rfds, NULL, NULL, &tv) > 0) {
84          if(FD_ISSET(0, &rfds)) { /* 標準入力から入力があったなら*/
85            if((nbytes = read(0, sbuf, sizeof(sbuf))) > 0){ /* 標準入力から読み込みクライアントに送信 */
86              write(csock, sbuf, nbytes);
87            }
88          }
89          if(FD_ISSET(csock, &rfds)) { /* ソケットから受信したなら*/
90          /* ソケットから読み込み端末に出力*/
91           if((nbytes = read(csock, cbuf, sizeof(cbuf))) > 0){
92              write(1, cbuf, nbytes);
93           }
94          }
95        }
96      } while (nbytes != 0);
97
98      /* read() が 0 を返すまで (=End-Of-File) 繰り返す */
99      close(csock);
100      printf("closed\n");
101    } while(1); /* 繰り返す*/
102  }
\end{verbatim}
}

\subsection{(課題2-2) simple\_talk\_client.c}
\label{sec:simple_talk_client_code}
\small{
\begin{verbatim}
 1  #include <sys/types.h>
 2  #include <sys/socket.h>
 3  #include <netinet/in.h>
 4  #include <netdb.h>
 5  #include <unistd.h>
 6  #include <string.h>
 7  #include <stdio.h>
 8  #include <stdlib.h>
 9  #define PORT 10130
10
11  int main(int argc, char **argv) {
12    int sock, nbytes;
13    char sbuf[1024], *name, cbuf[1024];
14    struct sockaddr_in svr;
15    struct hostent *hp;
16    fd_set rfds;
17    struct timeval tv;
18
19    if(argc != 2) {
20      fprintf(stderr, "Usage: %s hostname \n", argv[0]);
21      exit(1);
22    }else{
23      name = *(argv+1);
24    }
25
26    if((sock = socket(AF_INET, SOCK_STREAM, IPPROTO_TCP)) < 0){
27      perror("socket");
28      exit(1);
29    }
30
31    bzero(&svr, sizeof(svr));
32    svr.sin_family = AF_INET;
33    svr.sin_port = htons(PORT);
34    if((hp = gethostbyname(name)) == NULL){
35      perror("gethostbyname");
36      exit(1);
37    }
38    bcopy(hp->h_addr, &svr.sin_addr, hp->h_length);
39
40    if((connect(sock, (struct sockaddr *)&svr, sizeof(svr))) < 0){
41        perror("connect");
42        exit(1);
43    }
44    else {
45      printf("connected\n" );
46    }
47
48    do{
49      FD_ZERO(&rfds);
50      FD_SET(0, &rfds);
51      FD_SET(sock, &rfds);
52      tv.tv_sec = 1;
53      tv.tv_usec = 0;
54      memset(sbuf, '\0', sizeof(sbuf));
55      memset(cbuf, '\0', sizeof(cbuf));
56
57      if(select(sock+1, &rfds, NULL, NULL, &tv) > 0){
58        if(FD_ISSET(sock, &rfds)){
59          if((nbytes = read(sock, cbuf, sizeof(cbuf))) > 0){
60            write(1, cbuf, nbytes);
61          }
62        }
63        if(FD_ISSET(0, &rfds)){
64          if((nbytes = read(0, sbuf, sizeof(sbuf))) > 0){
65            write(sock, sbuf, nbytes);
66          }
67        }
68      }
69    } while(nbytes != 0);
70    close(sock);
71    printf("closed\n");
72  }
\end{verbatim}
}


\end{document}


%%%%%%%%%%% するべきこと %%%%%%%%%%%
% プログラムの最新版を引っ張ってくる.→済み
% Ctrl + Dで終了することを確認した実行結果をとる.→済み
% connectedと出力させる.→済み
% 各実行結果に行数を表示する.→済み
% 入力の途中に受診した時の実行結果.
% 仕様にエラーについても記述する.→無視
% %echo
% %simple_talk
